\documentclass{article}
\usepackage{graphicx} % Required for inserting images
\usepackage{tikz}
\usetikzlibrary{calc}
\title{Conlang test}
\author{Jarno Smets}
\date{February 2023}

\begin{document}

% - Setting the grid - %

% UpperLeft ------ CenterCeiling ------ UpperRight
% |											|
% | 											|
% | 											|
% | 											|
%CenterUpperLeft	   CenterUpper		   CenterUpperRight
% | 											|
% | 											|
% | 											|
% | 											|
%CenterLeft		     Center				CenterRight
% | 											|
% | 											|
% | 											|
% | 											|
%CenterBottomLeft  CenterBottom			CenterBottomRight
% | 											|
% | 											|
% | 											|
% | 											|
%BottomLeft	-----  CenterCellar	 -----	BottomRight										|




%Setting grid coordinates
\tikzset{grid/.pic={
    \node(UpperLeft) at (0, 10){};
    \node(UpperRight) at (7.5, 10){};
    \node(BottomRight) at (7.5, 0){};
    \node(BottomLeft) at (0,0){};
    \node(Center) at (3.75,5){};
    \node(CenterLeft) at (0,5){};
    \node(CenterRight) at (7.5,5){};
    \node(CenterUpper) at (3.75,7.5){};
    \node(CenterBottom) at (3.75,2.5){};
    \node(CenterCellar) at (3.75,0){};
    \node(CenterCeiling) at (3.75,10){};
    \node(CenterUpperLeft) at (0,7.5){};
    \node(CenterUpperRight) at (7.5,7.5){};
    \node(CenterBottomLeft) at (0,2.5){};
    \node(CenterBottomRight) at (7.5,2.5){};
    %Outside the grid
    \node(LeftFootout) at (-3.75, 0){};
	}
}

%Slanted grid

\def\slant{0.5}
\def\halfslant{0.5\slant}
\def\threequarterslant{0.75\slant}
\def\quarterslant{0.25\slant}
\tikzset{gridsl/.pic={
    \node(SLUpperLeft) at (0 + \slant , 10){};
    \node(SLUpperRight) at (7.5 + \slant , 10){};
    \node(SLBottomRight) at (7.5, 0){};%bottom - untouched - you can use BottomRight
    \node(SLBottomLeft) at (0,0){};%bottom - untouched - you can use BottomRight
    \node(SLCenter) at (3.75 + \halfslant ,5){};
    \node(SLCenterLeft) at (0 + \halfslant ,5){};
    \node(SLCenterRight) at (7.5 + \halfslant ,5){};
    \node(SLCenterUpper) at (3.75 + \threequarterslant  ,7.5){};
    \node(SLCenterBottom) at (3.75 + \threequarterslant ,2.5){};
    \node(SLCenterCellar) at (3.75,0){};%bottom - untouched - you can use BottomRight
    \node(SLCenterCeiling) at (3.75 + \slant,10){};
    \node(SLCenterUpperLeft) at (0 + \threequarterslant ,7.5){};
    \node(SLCenterUpperRight) at (7.5 + \threequarterslant ,7.5){};
    \node(SLCenterBottomLeft) at (0 + \quarterslant ,2.5){};
    \node(SLCenterBottomRight) at (7.5 + \quarterslant ,2.5){};
    %Outside the grid
    \node(LeftFootout) at (-3.75, 0){};
	}
}
%
%\newcommand*\insimg[1]{%
%\tikz[baseline=(key.base)]
%    \node[fill=white](test){#1\strut}
%} Code for inserting an image instead of text-token

%
%\tikzset{kut/.pic={
%    \draw[black, thick] (0,0) -- (0.3,0);
%    \draw[black,thick]  (0.15,0) -- (0.15, -0.3);
%}} Example for creating a symbol with coordinates

%% The thickness of the letters - since they are rectangles. These values are also embedded in \DeclareLetter
\def\backwardthickness{-0.1}
\def\forwardthickness{0.1}
\def\upwardthickness{0.1}
\def\downwardthickness{-0.1}

%Example 'T' letter
%\resizebox{8pt}{!}{
%\begin{tikzpicture}
%\pic{grid};
%\pic{gridsl};
%\draw[draw=black, fill=black] ($(UpperLeft) + (0,\downwardthickness)$) rectangle ($(UpperRight)+(0,\upwardthickness)$);
%\draw[fill=black] ($(CenterCeiling) + (\backwardthickness,0)$) rectangle ($(CenterCellar) + (\forwardthickness,0)$);
%\end{tikzpicture}
%}
test letter  %For reference

%Command for instantly designing the letter
%To be added: points and curves. White halos. This command should at one point give the letters their name. E.g.: \kut should be compiled as 'T' directly. 
\def\DeclareLetter#1#2#3#4{%
\def\backwardthickness{-0.1}
\def\forwardthickness{0.1}
\def\upwardthickness{0.1}
\def\downwardthickness{-0.1}
\def\corpsgrootte{8pt}
    \resizebox{\corpsgrootte}{!}{
        \begin{tikzpicture}
            \pic{grid};
            \pic{gridsl};
            \draw[fill=black] ($(#1) + (0,\downwardthickness)$) rectangle ($(#2)+(0,\upwardthickness)$);
            \draw[fill=black] ($(#3) + (\backwardthickness,0)$) rectangle ($(#4) + (\forwardthickness,0)$);
        \end{tikzpicture}
    }
}


\DeclareLetter{UpperLeft}{UpperRight}{UpperLeft}{BottomLeft} %Gamma -name?

\DeclareLetter{UpperLeft}{UpperRight}{UpperRight}{BottomRight} %Backwards Gamma - name?

\DeclareLetter{BottomLeft}{BottomRight}{CenterCeiling}{CenterCellar} %Upside-down T - name?

\DeclareLetter{BottomLeft}{BottomRight}{UpperRight}{BottomRight} %Backwards 'L' - name?

\DeclareLetter{BottomLeft}{BottomRight}{UpperLeft}{BottomLeft} % 'L' - name?

\DeclareLetter{CenterLeft}{CenterRight}{UpperLeft}{BottomLeft}
%single turnstile, kinda, name?

\DeclareLetter{CenterLeft}{CenterRight}{UpperRight}{BottomRight}
%reverse single turnstile, name?

\DeclareLetter{CenterLeft}{CenterRight}{CenterCeiling}{CenterCellar}
%Cross, name?

%\DeclareLetter{}

%%Slanted right-angles

\DeclareLetter{BottomLeft}{UpperRight}{UpperRight}{UpperLeft} 
%slanted T

\DeclareLetter{BottomLeft}{UpperRight}{UpperRight}{CenterCeiling}
%Slanted half-T

\DeclareLetter{BottomLeft}{UpperRight}{Center}{CenterLeft}
%Slanted reverse turnstile

\DeclareLetter{LeftFootout}{BottomLeft}{BottomLeft}{UpperRight}
%Reverse slanted L

\DeclareLetter{BottomLeft}{BottomRight}{CenterCellar}{UpperRight}
%Upside-down slanted T

\DeclareLetter{BottomLeft}{BottomRight}{BottomLeft}{UpperRight}
%Slanted L

\DeclareLetter{BottomLeft}{UpperLeft}{CenterLeft}{CenterRight}
%slanted single turnstile

\dotfill

\DeclareLetter{BottomLeft}{SLUpperLeft}{SLCenterLeft}{SLCenterRight}



\end{document}
